\documentclass[11pt,a4paper]{article}
\usepackage[utf8]{inputenc}
\usepackage[margin=1in]{geometry}
\usepackage{amsmath}
\usepackage{graphicx}
\usepackage{hyperref}
\usepackage{enumitem}
\usepackage{titlesec}
\usepackage{xcolor}
\usepackage{fancyhdr}
\usepackage{lastpage}

% Page formatting
\pagestyle{fancy}
\fancyhf{}
\fancyhead[L]{\textsc{Qwen3-VL Optimization Project}}
\fancyhead[R]{\textsc{20-Week Timeline}}
\fancyfoot[C]{\thepage\ of \pageref{LastPage}}

% Title formatting
\titleformat{\section}
{\Large\bfseries\color{blue!70!black}}
{}
{0em}
{}[\titlerule]

\titleformat{\subsection}
{\large\bfseries}
{}
{0em}
{}

% Hyperref setup
\hypersetup{
    colorlinks=true,
    linkcolor=blue,
    filecolor=magenta,
    urlcolor=cyan,
    pdftitle={Qwen3-VL Optimization Project Timeline},
    pdfauthor={Aditya Kothuri}
}

\title{\textbf{Optimizing Efficiency of the Qwen3-VL Vision-Language Model}\\
\large 20-Week Research Project Timeline}
\author{Aditya Kothuri}
\date{November 2025}

\begin{document}

\maketitle

\section{Project Overview}

\textbf{Objective:} Optimize Qwen3-VL's inference efficiency (latency, memory footprint, throughput) while preserving vision-language performance to enable real-time applications.

\textbf{Key Metrics:}
\begin{itemize}
    \item \textbf{Accuracy:} VQA accuracy, Image-captioning BLEU/ROUGE scores
    \item \textbf{Efficiency:} Inference latency (ms/example), Throughput (examples/sec), Memory usage (GPU/CPU RAM), Model size (MB)
\end{itemize}

\textbf{Optimization Techniques:}
\begin{itemize}
    \item Quantization (INT8/INT4)
    \item Structured and unstructured pruning
    \item Knowledge distillation
    \item Model architecture modifications
    \item Runtime optimizations (batching, kernel fusion, operator offloading)
\end{itemize}

\section{20-Week Timeline}

\subsection{Phase 1: Setup and Baseline Establishment (Weeks 1-4)}

\subsubsection*{Week 1: Environment Setup and Repository Preparation}
\begin{itemize}
    \item Set up development environment (Python, PyTorch, CUDA)
    \item Clone and configure Qwen3-VL repository
    \item Install dependencies and verify GPU access
    \item Set up version control and project structure
    \item Review Qwen3-VL architecture documentation
    \item Test basic inference with Qwen3-VL-2B model
    \item Document baseline system specifications
\end{itemize}

\subsubsection*{Week 2: Baseline Performance Measurement}
\begin{itemize}
    \item Implement baseline inference script for Qwen3-VL-2B
    \item Measure baseline metrics: latency, throughput, memory usage
    \item Profile model execution (identify bottlenecks)
    \item Test on sample VQA v2 dataset images
    \item Establish measurement methodology and logging system
    \item Create baseline performance report
    \item Set up automated benchmarking pipeline
\end{itemize}

\subsubsection*{Week 3: Dataset Preparation and Integration}
\begin{itemize}
    \item Download and preprocess VQA v2 dataset
    \item Create data loading pipeline for image-text pairs
    \item Implement evaluation metrics (VQA accuracy, BLEU, ROUGE)
    \item Create test/train splits for evaluation
    \item Validate dataset loading and preprocessing
    \item Document dataset statistics and characteristics
    \item Set up evaluation framework
\end{itemize}

\subsubsection*{Week 4: Baseline Accuracy Evaluation}
\begin{itemize}
    \item Run full baseline evaluation on VQA v2 test set
    \item Measure baseline accuracy metrics
    \item Generate baseline image-captioning outputs
    \item Create baseline performance dashboard
    \item Document baseline results comprehensively
    \item Identify challenging test cases for qualitative analysis
    \item Prepare baseline comparison template for future optimizations
\end{itemize}

\subsection{Phase 2: Initial Optimization Techniques (Weeks 5-8)}

\subsubsection*{Week 5: Quantization Implementation - INT8}
\begin{itemize}
    \item Research PyTorch quantization toolkit and Qwen3-VL compatibility
    \item Implement post-training quantization (PTQ) to INT8
    \item Test quantization on Qwen3-VL-2B model
    \item Measure efficiency gains (latency, memory, model size)
    \item Evaluate accuracy on VQA v2 validation set
    \item Document quantization configuration and results
    \item Identify quantization-sensitive layers
\end{itemize}

\subsubsection*{Week 6: Quantization Implementation - INT4 and Calibration}
\begin{itemize}
    \item Implement INT4 quantization with calibration dataset
    \item Test quantization-aware training (QAT) if applicable
    \item Compare INT8 vs INT4 trade-offs
    \item Measure accuracy degradation at different bit-widths
    \item Optimize calibration process
    \item Create quantization comparison report
    \item Select optimal quantization strategy for next phases
\end{itemize}

\subsubsection*{Week 7: Pruning Implementation - Unstructured}
\begin{itemize}
    \item Research pruning techniques for vision-language models
    \item Implement magnitude-based unstructured pruning
    \item Test different pruning ratios (10\%, 20\%, 30\%, 50\%)
    \item Measure sparsity vs accuracy trade-offs
    \item Evaluate efficiency gains from pruned models
    \item Document pruning methodology and hyperparameters
    \item Identify critical vs redundant weights
\end{itemize}

\subsubsection*{Week 8: Pruning Implementation - Structured}
\begin{itemize}
    \item Implement structured pruning (channel/head pruning)
    \item Test structured pruning on attention mechanisms
    \item Compare structured vs unstructured pruning results
    \item Measure hardware acceleration benefits of structured pruning
    \item Evaluate combined pruning strategies
    \item Create pruning comparison analysis
    \item Document best pruning approach for Qwen3-VL architecture
\end{itemize}

\subsection{Phase 3: Advanced Optimization Techniques (Weeks 9-12)}

\subsubsection*{Week 9: Knowledge Distillation Setup}
\begin{itemize}
    \item Research knowledge distillation for vision-language models
    \item Design student model architecture (smaller than 2B)
    \item Implement distillation loss functions (KL divergence, feature matching)
    \item Set up teacher-student training pipeline
    \item Prepare distillation training dataset
    \item Configure training hyperparameters
    \item Begin initial distillation experiments
\end{itemize}

\subsubsection*{Week 10: Knowledge Distillation Training}
\begin{itemize}
    \item Train student model using Qwen3-VL-2B as teacher
    \item Monitor training metrics and convergence
    \item Validate student model on VQA v2 validation set
    \item Compare student vs teacher performance
    \item Measure efficiency gains from smaller model
    \item Fine-tune distillation hyperparameters if needed
    \item Document distillation training process
\end{itemize}

\subsubsection*{Week 11: Architecture Modifications - Attention Mechanisms}
\begin{itemize}
    \item Research efficient attention mechanisms (Flash Attention, Sparse Attention)
    \item Implement efficient attention variants in Qwen3-VL
    \item Test different attention configurations
    \item Measure speedup from attention optimizations
    \item Evaluate accuracy impact of attention modifications
    \item Profile attention computation time
    \item Document attention optimization results
\end{itemize}

\subsubsection*{Week 12: Architecture Modifications - Layer Reduction}
\begin{itemize}
    \item Analyze transformer layer importance
    \item Implement layer removal/reduction strategies
    \item Test different layer configurations (remove 2, 4, 6 layers)
    \item Measure efficiency vs accuracy trade-offs
    \item Evaluate impact on different task types
    \item Create layer reduction analysis
    \item Document optimal layer configuration
\end{itemize}

\subsection{Phase 4: Runtime Optimizations (Weeks 13-16)}

\subsubsection*{Week 13: Batching and Throughput Optimization}
\begin{itemize}
    \item Implement dynamic batching strategies
    \item Optimize batch size for different input types
    \item Test batch processing efficiency
    \item Measure throughput improvements
    \item Evaluate memory usage with different batch sizes
    \item Optimize data loading pipeline
    \item Document batching optimization results
\end{itemize}

\subsubsection*{Week 14: Kernel Fusion and Operator Optimization}
\begin{itemize}
    \item Research PyTorch JIT compilation and TorchScript
    \item Implement operator fusion where applicable
    \item Test TensorRT or similar inference engines
    \item Profile operator-level performance
    \item Optimize critical computation paths
    \item Measure speedup from kernel optimizations
    \item Document operator-level optimizations
\end{itemize}

\subsubsection*{Week 15: Memory Optimization and Offloading}
\begin{itemize}
    \item Implement activation checkpointing
    \item Test CPU offloading for less critical operations
    \item Optimize memory allocation patterns
    \item Measure memory footprint reduction
    \item Evaluate impact on inference speed
    \item Test gradient checkpointing for training scenarios
    \item Document memory optimization strategies
\end{itemize}

\subsubsection*{Week 16: Integration and Combined Optimizations}
\begin{itemize}
    \item Combine best-performing optimization techniques
    \item Test compatibility between different optimizations
    \item Create optimized model variants (quantized + pruned, etc.)
    \item Measure cumulative efficiency gains
    \item Validate accuracy of combined optimizations
    \item Resolve any optimization conflicts
    \item Prepare final optimized model configurations
\end{itemize}

\subsection{Phase 5: Comprehensive Evaluation (Weeks 17-18)}

\subsubsection*{Week 17: Quantitative Evaluation}
\begin{itemize}
    \item Run comprehensive evaluation on all optimized variants
    \item Measure all metrics: accuracy, latency, throughput, memory, model size
    \item Generate comparison tables (baseline vs all variants)
    \item Create efficiency-accuracy trade-off curves
    \item Plot throughput vs model size graphs
    \item Analyze memory usage vs optimization level
    \item Document quantitative results comprehensively
\end{itemize}

\subsubsection*{Week 18: Qualitative Analysis and Case Studies}
\begin{itemize}
    \item Select representative test cases for qualitative analysis
    \item Compare outputs before and after optimization
    \item Identify failure cases and degradation patterns
    \item Document success cases where optimization maintains quality
    \item Create visualizations of model outputs
    \item Analyze error patterns and optimization impact
    \item Prepare qualitative analysis report with examples
\end{itemize}

\subsection{Phase 6: Documentation and Finalization (Weeks 19-20)}

\subsubsection*{Week 19: Results Compilation and Analysis}
\begin{itemize}
    \item Compile all experimental results
    \item Create comprehensive comparison tables
    \item Generate all plots and visualizations
    \item Write detailed methodology sections
    \item Analyze trends and patterns in results
    \item Identify best optimization strategies
    \item Prepare findings summary
\end{itemize}

\subsubsection*{Week 20: Final Documentation and Presentation}
\begin{itemize}
    \item Write final project report
    \item Create presentation materials
    \item Prepare code documentation and repository cleanup
    \item Create reproducibility guide
    \item Finalize all figures and tables
    \item Review and polish all documentation
    \item Prepare project deliverables
\end{itemize}

\section{Deliverables}

\begin{itemize}
    \item Optimized Qwen3-VL model variants with documented configurations
    \item Comprehensive evaluation report with quantitative and qualitative analysis
    \item Efficiency-accuracy trade-off analysis
    \item Comparison tables and visualizations
    \item Reproducible codebase with clear documentation
    \item Final project report and presentation
\end{itemize}

\section{Success Criteria}

\begin{itemize}
    \item Achieve measurable improvements in inference latency (target: 30-50\% reduction)
    \item Reduce memory footprint significantly (target: 40-60\% reduction)
    \item Maintain accuracy within 5\% of baseline on VQA v2
    \item Document all optimization techniques and their trade-offs
    \item Provide clear recommendations for real-time deployment scenarios
\end{itemize}

\section{Notes}

\begin{itemize}
    \item Timeline assumes 15-20 hours per week of focused work
    \item Weeks may be adjusted based on experimental results and challenges
    \item Regular checkpoints should be established to track progress
    \item Backup plans should be prepared for techniques that don't yield expected results
    \item Collaboration with advisors/peers for code review is recommended
\end{itemize}

\end{document}

